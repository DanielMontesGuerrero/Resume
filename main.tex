
%%%%%%%%%%%%%%%%%%%%%%%%%%%%%%%%%%%%%%%%%
% Developer CV
% LaTeX Template
% Version 1.0 (28/1/19)
%
% This template originates from:
% http://www.LaTeXTemplates.com
%
% Authors:
% Jan Vorisek (jan@vorisek.me)
% Based on a template by Jan Küster (info@jankuester.com)
% Modified for LaTeX Templates by Vel (vel@LaTeXTemplates.com)
%
% License:
% The MIT License (see included LICENSE file)
%
%%%%%%%%%%%%%%%%%%%%%%%%%%%%%%%%%%%%%%%%%

%----------------------------------------------------------------------------------------
%	PACKAGES AND OTHER DOCUMENT CONFIGURATIONS
%----------------------------------------------------------------------------------------

\documentclass[9pt]{developercv} % Default font size, values from 8-12pt are recommended

%----------------------------------------------------------------------------------------

\begin{document}

%----------------------------------------------------------------------------------------
%	TITLE AND CONTACT INFORMATION
%----------------------------------------------------------------------------------------

\begin{minipage}[t]{0.70\textwidth} % 45% of the page width for name
	\vspace{-\baselineskip} % Required for vertically aligning minipages
	
	% If your name is very short, use just one of the lines below
	% If your name is very long, reduce the font size or make the minipage wider and reduce the others proportionately
	\colorbox{black}{{\HUGE\textcolor{white}{\textbf{{Daniel}}}}} % First name
	
	\colorbox{black}{{\HUGE\textcolor{white}{\textbf{{MONTES GUERRERO}}}}} % Last name
	
	\vspace{6pt}
\end{minipage}
\begin{minipage}[t]{0.3\textwidth} % 27.5% of the page width for the first row of icons
	\vspace{-\baselineskip} % Required for vertically aligning minipages
	
	% The first parameter is the FontAwesome icon name, the second is the box size and the third is the text
	% Other icons can be found by referring to fontawesome.pdf (supplied with the template) and using the word after \fa in the command for the icon you wan
	\icon{Phone}{12}{+52 735 214 0767}\\
	\icon{Envelope}{12}{\href{mailto:dmontesguerrero17@gmail.com}{dmontesguerrero17@gmail.com}}\\
	\icon{Github}{12}{\href{https://github.com/DanielMontesGuerrero}{@DanielMontesGuerrero}}\\
\end{minipage}

%----------------------------------------------------------------------------------------
%	EDUCATION
%----------------------------------------------------------------------------------------

\cvsect{Education}

\begin{entrylist}
	\entry
		{2018-Present}
		{Bachelor’s degree in Computer Systems Engineering}
		{Graduating in December 2023}
		{\textbf{ESCOM-IPN (Superior School of Computer Sciences)}\\
		Student of academic excellence for 5 consecutive semesters.\\
		Participant of ``Algorithms Club - ESCOM''
		 \begin{flushright}
		 	CGPA 88 out of 100
		 \end{flushright}
		}
\end{entrylist}

%----------------------------------------------------------------------------------------
%	PROJECTS
%----------------------------------------------------------------------------------------

\cvsect{Projects}

\begin{entrylist}
	\entry
		{spring 2021}
		{\href{https://github.com/DanielMontesGuerrero/random-map-generator}{Random Map Generator}}
		{Programming language and compiler}
		{A language that allows you to describe 2D maps for videogames in Unity. The language implements algorithms to generate random maps using Perlin Noise. Offers ``smarttiles'' to autofill the map with tiles. A compiler was also built for the language to generate C\# scripts and use these scripts in Unity. Built in a team of three people.\\
		- Built with C\#, Unity, Flex and Bison.}
	\entry
		{}
		{Other projects}
		{}
		{
			- \href{https://github.com/DanielMontesGuerrero/sistema-picking-app}{Warehouse Resource Management System}\\
			- \href{https://github.com/DanielMontesGuerrero/la-tiendita}{Ecommerce website for students at ESCOM}\\
			- \href{https://github.com/DanielMontesGuerrero/anigitbot}{Discord bot to notify Github pull requests and issues}\\
			- \href{https://github.com/CatadoresDeAnime/potential-octo-meme}{Mobile game built with react native (Currentyl developing)}
		}
\end{entrylist}

\cvsect{Internships}

\begin{entrylist}
	\entry
		{summer 2022}
		{Microsoft}
		{Software Engineer Intern}
		{It was a 3 month internship. I created a tool for the business team to manage shipping promos for products in the Microsoft Store. This tool reduces the time required to create shipping campaings from 11 weeks to 2-3 weeks.}
	\entry
		{summer 2021}
		{Microsoft}
		{Explorer Intern}
		{It was a 3 month internship. I performed as a Program Manager the first month and as a Software Engineer the last two months. I worked with two other explorer interns in a project where we improved how configuration files are updated in the admin tool.\\
		- Improved UI and UX using AngularJS.
		- Worked with .NET for the backed.}
	\entry
		{fall 2021}
		{Meta}
		{Software Engineer Intern}
		{It was a 3 month internship. I worked in a dependency manager service to manage asynchronous workflows and their dependencies between Meta's internal services. I implemented methods to detect stuck workflows and workflows that were executed out of order.\\
		- Worked with c++ and Hack to add new health metrics for the service.\\
		- Used python to implement an out of order workflows detection tool.\\
		}
\end{entrylist}

%----------------------------------------------------------------------------------------
%	AWARDS
%----------------------------------------------------------------------------------------

\cvsect{Awards}

\begin{entrylist}
	\entry
		{1st place}
		{16th Annual Programming Contest ``ANIEI 2021'', team ``Discípulos de Marckess''}
		{2021}
		{}
	\entry
		{1st place}
		{XIV Annual Programming Contest ``Donald Knuth'' by ESCOM}
		{2022}
		{}
	\entry
		{7th place,\\out of 351 teams}
		{``Gran Premio de México'' - ICPC Programming Contest, team ``Cámara Ya Súbelo Asi w''}
		{2020}
		{}
\end{entrylist}

%----------------------------------------------------------------------------------------
%	ADDITIONAL INFORMATION
%----------------------------------------------------------------------------------------

\begin{minipage}[t]{0.25\textwidth}
	\vspace{-\baselineskip} % Required for vertically aligning minipages

	\cvsect{Technologies}
	
	\textbf{C++} - 3 years for competitive\\
	programming\\
	\textbf{Java} - 2 years\\
	\textbf{JavaScript} - 1.5 year\\
	\textbf{Python} - 1.5 year\\
	\textbf{C\#} - 1 year
\end{minipage}
\hfill
\begin{minipage}[t]{0.17\textwidth}
	\vspace{-\baselineskip} % Required for vertically aligning minipages
	
	\cvsect{Languages}
	
	\textbf{Spanish} - native\\
	\textbf{English} - TOEFL pBT
\end{minipage}
\hfill
\begin{minipage}[t]{0.48\textwidth}
	\vspace{-\baselineskip} % Required for vertically aligning minipages
	
	\cvsect{Activities}
	
	11th Argentina Training Camp of UNLaR - 2020\\
	1st Winter Training Camp of ESCOM Algorithm Club - 2020\\
	Summer Training Camp of ESCOM Algorithm Club - 2019
\end{minipage}

%----------------------------------------------------------------------------------------

\end{document}
